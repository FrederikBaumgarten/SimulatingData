\documentclass{article}

% required
\usepackage[hyphens]{url} % this wraps my URL versus letting it spill across the page, a bad habit LaTeX has
\usepackage{Sweave}
\usepackage{graphicx}
\usepackage{natbib}
\usepackage{amsmath}
\usepackage{textcomp}%among other things, it allows degrees C to be added
\usepackage{float}
\usepackage[utf8]{inputenc} % allow funny letters in citations 
\usepackage[nottoc]{tocbibind} %should add Re fences to the table of contents?
\usepackage{amsmath} % making nice equations 
\usepackage{listings} % add in stan code
\usepackage{xcolor}
\usepackage{capt-of}%allows me to set a caption for code in appendix 
\usepackage[export]{adjustbox} % adding a box around a map
\usepackage{lineno}
\linenumbers
% recommended! Uncomment the below line and change the path for your computer!
% \SweaveOpts{prefix.string=/Users/Lizzie/Documents/git/teaching/demoSweave/Fig.s/demoFig, eps=FALSE} 
%put your Fig.s in one place! Also, note that here 'Fig.s' is the folder and 'demoFig' is what each 
% Fig. produced will be titled plus its number or label (e.g., demoFig-nqpbetter.pdf')
% make your captioning look better
\usepackage[small]{caption}

\usepackage{xr-hyper} %refer to Fig.s in another document
\usepackage{hyperref}

\setlength{\captionmargin}{30pt}
\setlength{\abovecaptionskip}{0pt}
\setlength{\belowcaptionskip}{10pt}

% optional: muck with spacing
\topmargin -1.5cm        
\oddsidemargin 0.5cm   
\evensidemargin 0.5cm  % same as odd side margin but for left-hand pages
\textwidth 15.59cm
\textheight 21.94cm 
% \renewcommand{\baselinestretch}{1.5} % 1.5 lines between lines
\parindent 0pt		  % sets leading space for paragraphs
% optional: cute, fancy headers
\usepackage{fancyhdr}
\pagestyle{fancy}
%\fancyhead[LO]{Frederik Baumgarten}
%\fancyhead[RO]{Research Proposal}
% more optionals! %

\graphicspath{ {./figures/} }% tell latex where to find photos 

\begin{document}
	\renewcommand{\bibname}{References}%names reference list 
	
	%\SweaveOpts{concordance=TRUE} % For RStudio hiccups
	
	
	\title{1. The power of simulating data: a tool to design experiments, understand data limitations and improve scientific reasoning \\%(my favourite)
		
		%Alternate title ideas:
		2. Between noise and patterns: the power of data simulation in science to overcome perception biases\\
		3. The power of simulating data: a tool for scientists from designing experiments to drawing/reaching reasonable conclusions\\
		4. Between noise and patterns: Overcoming perception biases through data simulation
	} 
	
	
	\date{\today}
	\author{Frederik Baumgarten$^1$, Elizabeth Wolkovich, invite Andrew Gelman}
	\maketitle 
	
	\section*{Abstract}
	Advertise simulation (for all: frequentists, bayesian, machine learning)
	
	What's the Problem, what do we want to save and what's the solution\\
	- Werwolf: p-values, 'sloppy', simplified stats, over-interpretation of patterns that emerge by chance\\
	- Baby: scientific standards, correct conclusions, knowing the potential and limits of a dataset\\
	- Silverbullet: Data simulation through formulation of a mathematical model and playing with the parameters and the replication\\

	
	
	\section{Introduction}

			\subsection*{Opening example}
			Human desire for patterns even in pure noise \\
			Confirmation bias\\
			But noisy data\\
			how can we ensure a standard?\\
			\subsection*{Current solutions}
			scientific workflow (fig)\\
			include experiments, null hypothesis testing and their limits(not meaningull, not testable, not interesting) , analysis -> conclusions\\
			\subsection*{Growing evidence that this is not enough}
			p-values/replication crisis/overconfidence/overinterpretation/mathematical artefacts\\
			Show famous examples. -> could expand into box\\
			\subsection*{Some people hope to address this through machine learning}
			machines search for patterns without or less bias (or at least in a systematic/objective way)\\
			machine learning is usually amechanistic\\
			problem with hypothesis remain. searching for a model so it includes much of the assumptions/hypotheses expected from a useful model\\
			\subsection*{Aim}
			We propose an updated approach focussed on simulation + show how to do it\\
			Helps to address current gaps/limitations:\\
			build hypothesis, then formulation of mathematical model\\
			better design experiments\\
			avoid overconfidence/overinterpretation and mathematical artefacts\\
			

\section{Bus example with simulation workflow}
	
	\subsection*{Situation}
	model - show poisson distr.
	simulate
	\subsection*{no what?}
	still waiting for the bus...outlayer?\\
	update the model - assumptions\\
	add variable traffic jam\\
	
	
\section{Biological example - how to do it}
question based + we walk through each one.\\

			\subsection*{what influences y?}
			nitrogen
	
			\subsection*{what form? linear/nonlinear, near Gaussian, Poisson}
			linear
	
			\subsection*{What assumptions are reasonable?}
			for y, alpha, beta\\
			for effect sizes (parameters)\\
			for x data\\
			-> pick some for this example\\
	
			\subsection*{Simulate! }
			
	
\section*{How to use this - Play!}
so many ways, we highlight just a few 

			\subsection*{Power analysis}
			to better design experiments\\
	
			\subsection*{Avoiding overconfidence}
			play with replication while holding variance and effect size constant. p-value figure\\
	
\section*{Increasing importance in the future}

			\subsection*{Avoiding overconfidence}
			evergrowing Lit
			with AI we must learn to ask better questions\\
			the right questions and hypothesis that are testable with current + new methods. Build up house of knowledge instead of using new pattern finding algorithms\\
			how to integrate with AI?\\
	
\end{document}
